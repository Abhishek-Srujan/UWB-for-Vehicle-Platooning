This chapter explains about how the developed design method is validated and the results using that model is explained in the following sections. 
\section{Validation of the Design Method} 
The validation of design method is done using the reference plants. It is chosen in such a way that the reference plant is commercially well established and the data of the plant is available in the literature. Then the plant is designed using the developed model for the same specifications of the reference plant and it is compared against the reference plant to check the validity of the developed design method.  
\subsection{External receiver}
For the validation of external receiver, Gemasolar power plant in Spain is used as a test case. The developed design model is mainly aimed at molten salt receiver and so Gemasolar plant is the suitable test case. Gemasolar plant is the 20 MW solar tower power plant with 15 hours of storage located in the province of Seville, Spain. It is the first central tower plant with molten salt as a heat storage technology. The plant location and the characteristics are tabulated in the table 5.1.
\begin{table}[h!]
	\begin{center}
		\begin{tabular}{ |c|c|} 
			\hline
			\multicolumn{2}{|c|}{\textbf{Location}} \\
			\hline
			Latitude & 37.33°\\
			\hline
			Longitude & 5.19° \\
			\hline
			\multicolumn{2}{|c|}{\textbf{Plant Characteristics}} \\
			\hline
			Electric Power & 19.9 MWe \\
			\hline
			Storage Capacity & 15 hours \\
			\hline
			\multicolumn{2}{|c|}{\textbf{Receiver parameters}} \\
			\hline
			Receiver thermal power & 120 MWth \\
			\hline
			Receiver fluid & Molten salt \\
			\hline
			Tower height & 140 m \\
			\hline
		\end{tabular}
		\caption{Gemasolar power plant characteristics}
		\label{Gemasolar plant characteristics}
	\end{center}
\end{table}
\subsubsection{Receiver design}
The external receiver is designed using the model for the plant characteristics of Gemasolar and it is compared with the data available in the literature. 
\begin{table}[h!]
	\begin{center}
		\begin{tabular}{ |c|c|c|} 
			\hline
			\textbf{Design parameters} & \textbf{Gemasolar plant} & \textbf{devISEcrs model} \\
			\hline
			Receiver diameter & 8.1 m & 7.4 m\\
			\hline
			Receiver height & 10.6 m & 11 m \\
			\hline
			Diameter of receiver tube & 25 mm & 21 mm\\
			\hline
			Thickness of receiver tube & & 1.2 mm \\
			\hline
			No. of receiver panels & & 12\\
			\hline
			No. of tubes per panel & & 92\\
			\hline
			Tower height & 140 m  & \\
			\hline
		\end{tabular}
		\caption{Comparison of Gemasolar receiver design parameters}
		\label{Gemasolar plant characteristics}
	\end{center}
\end{table}
5.2 Receiver thermal losses

\subsection{Cavity receiver}
5.1 Receiver design
5.2 Receiver thermal losses
PS10 reference plant
Table of plant characteristics data and location
Table comparing design parameters comparison of receiver
Validation of the receiver efficiency and thermal losses
\section{Effect of Spillage Loss on Heliostat Size}