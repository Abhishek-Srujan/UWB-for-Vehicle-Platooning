\nomenclature[A]{SIC}{Successive Interference Cancellation}
\nomenclature[A]{BER}{Bit Error Rate}
\nomenclature[A]{UMTS}{Universal Mobile Telecommunication System}
\nomenclature[A]{CSMA}{Code Division Multiple Access}
\nomenclature[A]{EIRP}{Equivalent Isotropically Radiated Power}
\nomenclature[A]{SIR}{Signal to Interference Ratio}
\nomenclature[A]{DAA}{Detect and Avoid}
\nomenclature[A]{IVC}{Inter-vehicle Communication}

In Section 3.1, we present the related work about the use of UWB for inter-vehicle communication, interference of UWB to other narrowband communication protocols and vice-versa, and UWB Ranging. In Section 3.2, we highlight the novel aspects of the current work.

\section{Summary}
There have been several studies conducted on UWB radars for inter-vehicle communication, interference of UWB on other narrowband communication protocols and vice-versa, the coexistence of UWB with other narrowband communication protocols and UWB ranging as indicated below.

\subsection{UWB for inter-vehicle communication}

The authors of \cite{kohno2008latest} state that UWB is a promising technology for both communications and ranging in highly reliable systems such as ITS and medical healthcare. The authors of \cite{sakkila2008uwb} propose a road radar to be embedded on the vehicles based on UWB technology which provides good resolution and increased precision for detection and localization of obstacles, attributed to the high degree of accuracy that UWB provides and its capability to differentiate between various obstacles like cars, plates, pedestrians, etc. The authors of \cite{doi2004frequency} propose an inter-vehicle UWB radar system using chirp waveforms instead of the conventional UWB-IR system that uses a modulated Gaussian pulse wherein the transmitted signal consists of a linear combination of chirp signals with the same time duration but different frequency bands. They show that this system has a lower non-detected rate than the conventional UWB-IR system, and hence suitable for ITS applications. 

The authors of \cite{takahara2012study} propose a UWB radar system assisted by communication to improve safety by obtaining additional information of the target vehicle by communication. They show that it is possible to improve the precision of ranging against the distant vehicles by following this approach through computer simulations. In the proposed scheme, the data for communication is appended to the UWB radar signal and the preceding target vehicle that receives the radar signal obtains information of the following vehicle. Moreover, the target vehicle transmits its driving information to the following vehicle after specific time duration. The following vehicle receives not only the reflected radar signal but also the following communication signal from the target vehicle. By using these two signals, more precise ranging and information collection can be performed. They indicate that in a conventional UWB radar, the precision of ranging usually degrades against distant targets because of degradation of the signal to noise ratio of the reflected signal from the target vehicle. However, using the proposed scheme, it is possible to improve ranging precision by detecting the communication signal that is transmitted just after the reflected signal from the target vehicle. 

In \cite{takahara2013study}, the authors discuss the problems encountered as a result of the proposed system presented above. They state that near-far problem due to the difference of signal power from each vehicle and inter-symbol interference while communicating with multiple devices was encountered and propose an iterative detection system using Successive Interference Cancellation (SIC) to improve the detection rate of the signal and the bit error rate (BER) performance. In the iterative detection system, the signals except the desired signal detected by SIC is subtracted from the original received signal leading to improved performance in simulations. The authors of \cite{takahara2014study} determine that the data to be transmitted through communication should include vehicle ID and the transmission time of the signal.

The authors of \cite{elbahhar2005using} propose an inter-vehicle communication system based on UWB. They study two types of UWB waveforms: coded Gaussian and monocycle pulses and state that Gaussian pulses waveform was the better of the two since the obtained BER was lower than that for monocycle pulses during simulations.

\subsection{Interference of UWB on narrowband communication and vice-versa}
In \cite{hamalainen2003ultra}, the authors investigated the level of impact of UWB devices on IEEE 802.11b and Bluetooth networks using a UWB device corresponding to hundreds of FCC-compliant UWB devices because of its high transmitted power level, in the 2.4-GHz ISM band. They state that under the extreme interference conditions examined, the UWB devices can have an impact on both IEEE 802.11b and Bluetooth networks, depending on the separation from the victim system. For interference distances of less than 50 cm, the UWB interferers affected the reported SNR for both LOS and NLOS cases. The worst case degradation of the received SNR in the IEEE 802.11b was less than 15 dB for 20 UWB devices (equivalent to several thousand FCC-complaint UWB devices) at 10 cm distance. A corresponding drop in the network throughput was observed only for the NLOS case and only for distances of less than 35 cm. In the LOS case, the impact of the UWB devices was insignificant. With respect to the Bluetooth connection, they state that it does not suffer significantly from the UWB interferers. The resulting decrease in throughput was approximately 20 Kbps in the worst case.

The authors of \cite{sadowski2011narrowband} measured the narrowband transmission quality in the presence of impulse radio UWB interference with unmodified IEEE 802.15.4a and modified IEEE 802.15.4a IR UWB signal. They state that selective reduction of UWB power spectral density at the frequency of narrowband transmission allowed to reduce bit error rate without the need to decrease the power of UWB interferer. 

In \cite{findikli2011performance}, the authors assess the UWB-IR system performance in the presence of an active narrow-band system. They show that while the BER performances of coherent and non-coherent receiving structures may be slightly degraded with the use of a linear combination of pulses when there is no active narrow-band system, the performances can be significantly improved with appropriate filtering techniques at the receiver when a narrow-band system is active. The authors of \cite{taha2002theoretical} state that interference coming from external sources degrade the UWB radio performance and depending on the frequency of narrowband interferers, they degrade the performance differently. They state that those narrow band interferers with frequencies concentrated at regions where the UWB radio pulse has stronger frequency contents, degrade the performance more severely. Moreover, they state that careful design of UWB pulse shape can mitigate the narrowband interference.

In \cite{ahmed2008impact}, the authors assess the effect of UWB on the Universal Mobile Telecommunication System (UMTS) and Code Division Multiple Access systems (CSMA-450). They show that, for the case of a single UWB transmitter, the UMTS can easily tolerate UWB interference when the UWB Equivalent Isotropically Radiated Power (EIRP) is -92.5 dBm/MHz or less for a distance between the UWB transmitter and the UMTS mobile of 1 m or higher. Also, they show that, for the case of multi-UWB transmitters, the UMTS can easily tolerate the UWB interference when the UWB EIRP is -94.5 dBm/MHz. For the single UWB transmitter case, the CDMA-450 downlink can tolerate UWB interference when the UWB power density is in the order of -106 dBm/MHz. For the case of multi-UWB transmitters, the power density that can be tolerated by the downlink of the CDMA-450 system is in the order of -108 dBm/MHz.

In \cite{chiani2009coexistence}, the authors assess the coexistence between UWB and narrowband wireless communication systems. Concerning UWB systems affected by narrowband interference, they show that the impact of narrowband interference strongly depends on, for a UWB-IR coherent receiver, on the carrier frequency of the interferer, the UWB pulse shape, and the spreading code adopted. They found that there was no significant performance degradation in the UWB link for Signal to Interference Ratio (SIR) on the order of -20 dB or greater. However, they state that the low transmitted power level currently allowed for UWB systems, which is typically much lower than that for narrowband transmitters could lead to a scenario wherein strong narrowband interferers could produce very small SIR at the UWB receiver. Similarly, for the dual case of narrowband systems affected by UWB interference, they show that the effects of a single UWB interferer are almost negligible, and the performance of the narrowband links are practically unchanged for sufficiently large SIR values. However, they state that in situations where the narrowband receiver is much closer to the UWB transmitter than to narrowband transmitter, can lead to very low SIR, with a consequent performance degradation in the narrowband link.

The authors of \cite{lewandowski2010coexistence} examine the coexistence of 802.11b and 802.15.4a-CSS through simulations and show that an error free coexistence of IEEE 802.11b and IEEE 802.15.4a-CSS is not feasible. The authors of \cite{mishra2007detect} show that in their UWB/WiMax coexistence experiments conducted by using a single UWB device with WiMax system, it is indeed feasible for UWB and WiMax system to co-exist by using Detect and Avoid (DAA) mechanism.

\subsection{UWB Ranging}

In \cite{soganci2011accurate}, the authors observe that time-based ranging is well suited for UWB systems. In \cite{lanzisera2009rf}, the authors tested outdoor ranging using two Waldo nodes in a parking lot with some cars but mostly open space. The environment was providing a baseline for ranging performance in an atmosphere where there is relatively little multipath interference. The method of communication between the nodes was through a wireless link. Range estimates were taken at distances ranging from 1 m to 45 m, and the received signal strength estimates were considered as well. They found that the received signal strength range estimates did not provide good range estimates compared to the time of flight estimates even in a mild multipath environment, but the time of flight measurements performed much better. They found that approximately 80\% of the time of the flight measurements were accurate to within 1 m, but not even 20\% of the received signal strength based estimates were within 1m. The authors of \cite{kristem2014experimental} carried out UWB ranging measurement in a dense urban environment at distances of 20 m,30 m, and 40 m, with two different antenna heights. They observed that the root mean square error in the range increased from 0.12 m to 0.14 m for LOS measurements and from 7.9 m to 9.8 m in NLOS measurements when the antenna height reduced from 100 cm to 10 cm.

The authors of \cite{petovello2012demonstration} propose the concept of differential GPS relative navigation augmented with UWB and bearing measurements. They found that combining GPS pseudo range, UWB range, and bearing measurements can significantly improve horizontal positioning accuracy, particularly in environments where GPS availability is reduced. The UWB measurements contributed to an improved along-track relative position while the bearing measurements improved the across-track position. 

The authors of \cite{ye2011experimental} examine the effect of LOS and NLOS ranging in indoor and outdoor environments using IEEE 802.15.4a impulse UWB transceiver. They found that the average ranging error was less than 20 cm when measured outdoors for a distance of 10 m in LOS. In the NLOS case, they found that the ranging error varied from 6 cm (Glass) to 29 cm (Door) when measured with a 3 m testing point.

\section{Novelty}
In this Section, we describe the limitations of the existing work and the novel aspects of our work in the field of inter-vehicle communication.

\subsubsection{Limitations}
\begin{enumerate}
	
	\item Interference of UWB to WiFi-p or vice-versa has not been studied.
	\item The influence of Channel, Preamble length, PRF, Standard SFD, Non-standard SFD, Smart Power enablement, Preamble code on UWB link in the presence of WiFi-p interference has not been studied.
	\item The feasibility of using UWB as a secondary independent backup channel for platooning has not been studied.
	\item The performance of IR-UWB link in the presence of speed or acceleration difference has not been documented.
	\item Effect of speed or acceleration on ranging accuracy with respect to platooning has not been documented.
	\item Existence of UWB Nulls has not been documented.
	\item Effect of Standard SFD and Non-standard SFD on ranging accuracy has not been documented.
	\item UWB for Inter-Vehicle Communication (IVC) has been evaluated only through UWB radars and simulations. UWB communication for IVC through actual experiments has not been documented.
\end{enumerate}

IEEE 802.15.4a is a relatively new standard, which has not been tested in an inter-vehicle and outdoor environment for platooning. This thesis focuses on handling all the above limitations except limitation 5. We design a set of measurement goals in Chapter 4 to study the feasibility of using UWB as a secondary independent backup channel for platooning. We implement a test platform (Chapter 5) to study the measurement goals in different test set-ups. We study the LDC requirements of UWB in (Chapter 6). We conduct experiments (Chapter 7) in various test set-ups under different experimental scenarios and analyze the outcomes to study the above-mentioned measurement goals.

