Highly Automated Driving is experiencing a huge interest worldwide from many stakeholders. It holds the promise of increasing safety, comfort, personal productivity or entertainment, traffic efficiency, cost and environmental benefits. One area of investigation is the ability of cars to drive closer together, or even to platoon. Since physical sensors such as radars and cameras do not provide the required system performance, reliable communication between vehicles is a means to achieve this. While the wireless communication standard ITS-G5 is being selected as the technology of choice, ultra - wideband (UWB) as a second independent communication technology may provide additional benefits. UWB physical layer enables accurate ranging at low latency, improved robustness against interference, ultra low power consumption, low power spectral density, low cost, low impact on existing narrowband systems and lower probability of intercept and detection.

In this thesis, we investigate the feasibility of using UWB next to WiFi-p, examine the robustness of UWB communication in platooning mode with respect to differences in velocities, typical inter-vehicle environment, outdoor environment and ascertain the ranging accuracy that can be achieved through UWB. We implemented a test platform, TP-UWB (Test Platform for evaluation and analysis of UWB) using LPCXpresso 4337 and DecaWave DW1000 UWB transceiver. The test platform enables us to measure the UWB link quality and range, perform test automation by varying test parameters that influence UWB behavior such as packets or message, channels, Pulse Repetition Frequency (PRF), preamble lengths, preamble codes, standard or non-standard Start of the Frame Delimiter (SFD) and Smart Power enablement. We conduct experiments in table-top, outdoor and inter-vehicle environments using the test platform. We then perform a behavioral analysis of UWB using different parameters indicated above and determine the best configuration of UWB for platooning such that it operates with least possible interference to WiFi-p.

Conclusion not included for reasons of confidentiality.

\textbf{Keywords}: Platooning, UWB, WiFi -p, Interference, Range, Ranging.  